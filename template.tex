\documentclass[10pt,twocolumn]{article} 

% use the oxycomps style file
\usepackage{oxycomps}

% read references.bib for the bibtex data
\bibliography{references}

% include metadata in the generated pdf file
\pdfinfo{
    /Title (Ethical Considerations of Barbell Velocity and Path Tracking Mobile App)
    /Author (Neython Lec Streitz)
}

% set the title and author information
\title{Ethical Considerations of Barbell Velocity and Path Tracking Mobile App}
\author{Neython Lec Streitz}
\affiliation{Occidental College}

\email{nlecstreitz@oxy.edu}

\begin{document}

\maketitle

\section{Ethical Considerations}
At first glance, a mobile application for providing velocity and bar path tracking for barbell-based exercises is innocuous. However, there exists a handful of ethical considerations to be weary of, especially with the resources currently available for this project. These issues are directly linked to the main goals of the project. That is, providing barbell tracking accurately, increasing accessibility by lowering associated costs, and allowing tracking to occur over time. These associated ethical considerations thus include: the potential of providing faulty tracking to users, making barbell exercises less accessible/inviting via over-complication, accessibility issues with app design/features, and privacy concerns with maintaining user videos and metrics. Individually, these potential issues can be addressed, but as a whole, they might entail spending too much time on disparate features. Moreso, the current development plan of the project involves building solely the velocity and path tracking model before implementing it in an app. This means that ethical issues with app design likely will not be a main area of concern. Given that the ethical issues are closely tied to the fundamental goals of the project, we believe that this project should not continue in production without serious consideration into an ethically sound implementation.

\subsection{Providing Accurate Velocity}
Paramount to the utility of the project is the need for accurate tracking of bar path and bar velocity. Inherent with VBT methods is the fact that slight variances in velocity indicate important information about neuromusclar and functional performance \cite{Dorrell2020}. It is these changes that velocity tracking hopes to analyze. Thus, one of the largest potential issues with the project is that large enough deviations from true velocity scores have the potential to invalidate the training method.  \par

While deviations on the lower end might only lead to under-prescribing lifting intensity, deviations on the higher end have the potential to injure lifters relying on the application. When speaking on the ethical considerations of AI-based health-related apps, researcher Michael Kühler argues that "AI health apps... arguably lead to the novel issue of AI paternalism, particularly in the health care domain". That is, health and fitness apps are an instance of "persuasive technology" whereby the app relies on strategies (gamification, positive feedback, etc.) to motivate users towards specific ends \cite{Kuhler2022}. The largest criticism of paternalism is the idea that it relies on an external notion of what is good for a person. A main advantage of VBT over traditional percentage-based training is the added personalizability that comes with basing training on daily and historic metrics. Nonetheless, this added personalizability gives VBT the allure of objectivity when it comes to prescribing exercise volume and load, meaning dedicated lifters or even novices might force themselves to perform more than they can handle. This can lead to problems of injury, overtraining, or burn-out, all issues that are contrary to the stated goal of the app.\par

The fact of the matter is that these issues do not exist if the app can provide sufficient accuracy as well as present the information in a way that maintains a user's autonomy. However, the idea of providing necessary accuracy is much easier said than done. Currently, there exists no mobile applications on the market that have been fully validated by existing scientific literature: "While there is conflicting evidence, it appears that substantial bias and error can be introduced when different devices and/or users implement these measuring tools" \cite{Weakley2021}. The PowerLift app has shown some promise in a handful of studies, however, other researchers have come out against those studies citing improper validation and conflicts of interest \cite{Courel-Ibanez2020}. The handful of apps studied in the literature all have more resources available to them than what we have for this project, and yet they struggle to provide enough accuracy to make their apps worthwhile \cite{Martinez-Cava2020}. Accuracy is exceptionally vulnerable to slower tempo lifts, which some users might program intentionally: "An assumption with velocity-based training is that you move each rep as fast as you can. If you don’t, your velocity data is essentially worthless, since all of the ways you can prescribe training using velocity is predicated on the linear relationship between load and velocity" \cite{Nuckols2019}. Therefore, it is unlikely that our project will provide accurate enough velocity metrics to be reliably used for comprehensive velocity-based programming. This means that the output of the project either needs to be framed in a different way (as a reference, not a prescription) or the focus needs to be on only the bar path component, or both. Otherwise, our project is essentially lying to it users if we market it as a tool for implementing VBT.

\subsection{Approachability and Accessibility of Lifting}
Beyond the technical aspects of the project, there are issues with the project's goal of increasing the accessibility of VBT for the general population. While the idea of the project is to provide a low-cost/practically free method of velocity tracking, this is assuming a person already has access to a barbell, weights, and importantly, an iPhone. Even with these requirements in place, a conflict exists between selling the app via the studied benefits of VBT and acknowledging the limitations and often impracticality of VBT for most lifters. Leaning too forwards into selling VBT might push novice lifters away from resistance training because of a fear of complexity. Furthermore, the bar path tracking feature has the potential to reinforce body type and body movement standards, not based on a wide variety of people. \par

Firstly, VBT as a philosophy of training is a relatively emerging field. Like most things that are the latest-and-greatest, many proponents of VBT laud it as the only answer to resistance training programming. The development of this app is in line with the goals of this community, however, in doing so, it contributes to the one-and-only mindset. That is, in order to market the app successfully we must explain the advantages of VBT over traditional percentage-based programming. Depending on the framing, those explanations have the potential to put down percentage-based training, which has a long and studied history of being a valid method of programming. Why is this an issue? For one, it might push people interested in resistance training away from pursuing it, out of a fear that it is too complex. The fact of the matter is that VBT is an advanced method for tracking and planning training. Thus, by presenting it as the one-and-only solution, it might force people into a mindset where they either follow it or do not attempt barbell-based exercises at all. It is of upmost importance that the framing of this app pays close attention to how VBT is discussed, including the benefits and downsides, as well as specifies the target audience as intermediate to advanced lifters. Education novice lifters on how to use velocity and bar path tracking would go a long way in alleviating this potential issue, yet this might simply be out of scope for the project. \par

Secondly, complementary to the barbell path tracking feature is a guide for the optimal bar path. The idea here is to provide a guiding line to show lifter's where they might improve in terms of the way they move the barbell throughout a lift. Compared to velocity tracking, barbell path tracking is a much simpler and beginner-friendly tool. The problem here is that this requires us to specify an optimal bar path. While there does exist cues and guidelines for the way each exercise should be performed, these can easily be invalid for people looking to bias different body parts, with different body leverages and biomechanics, and overall with just other goals than complete optimization of their bar path. Importantly, providing an optimized and individualized bar path for a person's specific body is likely near impossible for the scope of this project. Thus, having only one standard bar path, almost certainly based on a person who is able-bodied, has the potential to ostracize people with different body types or people who are disabled. In a study on the barriers and facilitators of exercise, Nikolajsen et al., identified a lack of tailored/adaptive exercise programs as a primary barrier of entry for many people looking to start exercising \cite{Nikolajsen2021}. As mentioned before, as an example of persuasive technology, our app could lead people to look down on themselves for not matching the optimized bar path, as well as reinforce a quantified notion of how a person should move based only on a specific population of people. \par

\subsection{User Experience and Accessibility of the App}
As an app, care must be taken in ensuring an accessible user experience for all people. In their paper on accessibility issues for android apps, Alshayban et al., describe common accessibility problems associated with limited content labeling, UI implementation, and touch target size \cite{Alshayban2020}. These same issues are quite apparent with the current potential design of the app. \par

For one, the bar path tracking component of the app is a visual experience. Unfortunately, this means that people who are blind/low-vision (BLV) are unlikely to benefit from that feature. However, this doesn't mean that they can't benefit from the velocity tracking component. Currently, the project has no plans to produce audio feedback during a set. Yet, this would aid not only in making the app more accessible, but even for people who are not BLV, might benefit them so they do not have to look over during a set and now receive in-set recommendations \cite{Dorrell2020}. With that in mind, the app would have to be designed in a way that would allow people who are BLV to access the velocity features without much trouble. A complete audio-only experience is likely to be out of the scope for this project, so in that way, our app suffers from issues of accessibility. \par

Furthermore, people with motor control issues might have difficulty navigating the app, especially features such as scrolling through a video to watch the bar path or inputting information about exercises done. Fortunately, increasing touch target sizes and staying away from certain UI implementations mentioned by Alshayban et al. is not too difficult. Color contrast is also something that is easily implemented while also helpful to those with visual impairments. With that said, the majority of participants for any user-testing of this app are unlikely to be those with disabilities. It is quite probable that the people included in the testing of the app are going to be college athletes. Thus, even though common accessibility issues are likely and easily addressed, there is a large potential for the development to miss items that we simply do not think to include/modify.

\subsection{Privacy of Health Data}
Certainly, seeing bar path and velocity metrics on a given day allows lifters and coaches to modify and improve their daily resistance training programming. Even more useful is tracking those metrics across time, as a lifter/coach gains an understanding of a lifter's progression, performance compared to previous sessions, and patterns in training. So, our app aims to deliver a feature based on providing and comparing metrics across time. However, this comes with its own issues and it now requires the app to store user data. This might include videos of exercises being performed, velocity metrics per day, or simplified bar path graphs. As a result, there are new ethical considerations in terms of privacy concerns. \par

As Kühler points out "even if AI health apps’ paternalistic influence on users were beneficial for them and did not undermine or diminish their autonomy, such apps might still end up being considered ethically problematic" \cite{Kuhler2022}. Barbell tracking definitely would stand to improve by tapping into user data. As the app would store the videos and metrics of our users, there is a potential for a substantial repository of data to pull from. Yet, this is exactly the issue with storing data, specifically, health and fitness data. As an app, users need not just be people interested in weight lifting. Perhaps, a user might be a physical therapist using the app to help their patients improve their movement. This would truly mean our app is storing health data. As a result, using user data requires informed consent and opacity on the side of the developers. Less outwardly questionable is the issue of simply keeping the data private from external sources. To mitigate this potential problem, Schaar recommends employing, \textit{Privacy by Design} including minimizing data collection, data sovereignty, and right to information among a handful of other concerns \cite{Schaar2010}. There does not seem to appear any potential features of the app that would deny these recommendations, but it will lead to longer development time. This is especially true since the projected development of the app starts not by app design but by focusing on creating accurate bar path and velocity tracking. Regardless, storing health data of app user's will necessitate careful consideration in terms of data usage and data privacy.

\subsection{Conclusion}
All in all, we've identified several potential sources of ethical drawbacks to the project. Firstly, we acknowledge how marketing the app as an upgrade to traditional percentage-based training might scare off potential beginner's to resistance training. Alongside, components of the app might reinforce body type and body movement standards based on a limited set of potential users. Furthermore, we've identified potential problems with the app design and app features including issues of accessibility for certain populations and issues of privacy and usage for user data. As a result, we believe that a re-framing of the app's goals is necessary if the project is to continue. Given the focus on the velocity and bar path tracking feature, there is a serious potential for lacking app design. In order for this project to truly aid in the availability of velocity-based training for people that resistance train, the entire experience must be accessible, not simply the cost.

\printbibliography 

\end{document}
